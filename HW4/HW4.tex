% Options for packages loaded elsewhere
\PassOptionsToPackage{unicode}{hyperref}
\PassOptionsToPackage{hyphens}{url}
\PassOptionsToPackage{dvipsnames,svgnames,x11names}{xcolor}
%
\documentclass[
  letterpaper,
  DIV=11,
  numbers=noendperiod]{scrartcl}

\usepackage{amsmath,amssymb}
\usepackage{iftex}
\ifPDFTeX
  \usepackage[T1]{fontenc}
  \usepackage[utf8]{inputenc}
  \usepackage{textcomp} % provide euro and other symbols
\else % if luatex or xetex
  \usepackage{unicode-math}
  \defaultfontfeatures{Scale=MatchLowercase}
  \defaultfontfeatures[\rmfamily]{Ligatures=TeX,Scale=1}
\fi
\usepackage{lmodern}
\ifPDFTeX\else  
    % xetex/luatex font selection
\fi
% Use upquote if available, for straight quotes in verbatim environments
\IfFileExists{upquote.sty}{\usepackage{upquote}}{}
\IfFileExists{microtype.sty}{% use microtype if available
  \usepackage[]{microtype}
  \UseMicrotypeSet[protrusion]{basicmath} % disable protrusion for tt fonts
}{}
\makeatletter
\@ifundefined{KOMAClassName}{% if non-KOMA class
  \IfFileExists{parskip.sty}{%
    \usepackage{parskip}
  }{% else
    \setlength{\parindent}{0pt}
    \setlength{\parskip}{6pt plus 2pt minus 1pt}}
}{% if KOMA class
  \KOMAoptions{parskip=half}}
\makeatother
\usepackage{xcolor}
\setlength{\emergencystretch}{3em} % prevent overfull lines
\setcounter{secnumdepth}{-\maxdimen} % remove section numbering
% Make \paragraph and \subparagraph free-standing
\ifx\paragraph\undefined\else
  \let\oldparagraph\paragraph
  \renewcommand{\paragraph}[1]{\oldparagraph{#1}\mbox{}}
\fi
\ifx\subparagraph\undefined\else
  \let\oldsubparagraph\subparagraph
  \renewcommand{\subparagraph}[1]{\oldsubparagraph{#1}\mbox{}}
\fi

\usepackage{color}
\usepackage{fancyvrb}
\newcommand{\VerbBar}{|}
\newcommand{\VERB}{\Verb[commandchars=\\\{\}]}
\DefineVerbatimEnvironment{Highlighting}{Verbatim}{commandchars=\\\{\}}
% Add ',fontsize=\small' for more characters per line
\usepackage{framed}
\definecolor{shadecolor}{RGB}{241,243,245}
\newenvironment{Shaded}{\begin{snugshade}}{\end{snugshade}}
\newcommand{\AlertTok}[1]{\textcolor[rgb]{0.68,0.00,0.00}{#1}}
\newcommand{\AnnotationTok}[1]{\textcolor[rgb]{0.37,0.37,0.37}{#1}}
\newcommand{\AttributeTok}[1]{\textcolor[rgb]{0.40,0.45,0.13}{#1}}
\newcommand{\BaseNTok}[1]{\textcolor[rgb]{0.68,0.00,0.00}{#1}}
\newcommand{\BuiltInTok}[1]{\textcolor[rgb]{0.00,0.23,0.31}{#1}}
\newcommand{\CharTok}[1]{\textcolor[rgb]{0.13,0.47,0.30}{#1}}
\newcommand{\CommentTok}[1]{\textcolor[rgb]{0.37,0.37,0.37}{#1}}
\newcommand{\CommentVarTok}[1]{\textcolor[rgb]{0.37,0.37,0.37}{\textit{#1}}}
\newcommand{\ConstantTok}[1]{\textcolor[rgb]{0.56,0.35,0.01}{#1}}
\newcommand{\ControlFlowTok}[1]{\textcolor[rgb]{0.00,0.23,0.31}{#1}}
\newcommand{\DataTypeTok}[1]{\textcolor[rgb]{0.68,0.00,0.00}{#1}}
\newcommand{\DecValTok}[1]{\textcolor[rgb]{0.68,0.00,0.00}{#1}}
\newcommand{\DocumentationTok}[1]{\textcolor[rgb]{0.37,0.37,0.37}{\textit{#1}}}
\newcommand{\ErrorTok}[1]{\textcolor[rgb]{0.68,0.00,0.00}{#1}}
\newcommand{\ExtensionTok}[1]{\textcolor[rgb]{0.00,0.23,0.31}{#1}}
\newcommand{\FloatTok}[1]{\textcolor[rgb]{0.68,0.00,0.00}{#1}}
\newcommand{\FunctionTok}[1]{\textcolor[rgb]{0.28,0.35,0.67}{#1}}
\newcommand{\ImportTok}[1]{\textcolor[rgb]{0.00,0.46,0.62}{#1}}
\newcommand{\InformationTok}[1]{\textcolor[rgb]{0.37,0.37,0.37}{#1}}
\newcommand{\KeywordTok}[1]{\textcolor[rgb]{0.00,0.23,0.31}{#1}}
\newcommand{\NormalTok}[1]{\textcolor[rgb]{0.00,0.23,0.31}{#1}}
\newcommand{\OperatorTok}[1]{\textcolor[rgb]{0.37,0.37,0.37}{#1}}
\newcommand{\OtherTok}[1]{\textcolor[rgb]{0.00,0.23,0.31}{#1}}
\newcommand{\PreprocessorTok}[1]{\textcolor[rgb]{0.68,0.00,0.00}{#1}}
\newcommand{\RegionMarkerTok}[1]{\textcolor[rgb]{0.00,0.23,0.31}{#1}}
\newcommand{\SpecialCharTok}[1]{\textcolor[rgb]{0.37,0.37,0.37}{#1}}
\newcommand{\SpecialStringTok}[1]{\textcolor[rgb]{0.13,0.47,0.30}{#1}}
\newcommand{\StringTok}[1]{\textcolor[rgb]{0.13,0.47,0.30}{#1}}
\newcommand{\VariableTok}[1]{\textcolor[rgb]{0.07,0.07,0.07}{#1}}
\newcommand{\VerbatimStringTok}[1]{\textcolor[rgb]{0.13,0.47,0.30}{#1}}
\newcommand{\WarningTok}[1]{\textcolor[rgb]{0.37,0.37,0.37}{\textit{#1}}}

\providecommand{\tightlist}{%
  \setlength{\itemsep}{0pt}\setlength{\parskip}{0pt}}\usepackage{longtable,booktabs,array}
\usepackage{calc} % for calculating minipage widths
% Correct order of tables after \paragraph or \subparagraph
\usepackage{etoolbox}
\makeatletter
\patchcmd\longtable{\par}{\if@noskipsec\mbox{}\fi\par}{}{}
\makeatother
% Allow footnotes in longtable head/foot
\IfFileExists{footnotehyper.sty}{\usepackage{footnotehyper}}{\usepackage{footnote}}
\makesavenoteenv{longtable}
\usepackage{graphicx}
\makeatletter
\def\maxwidth{\ifdim\Gin@nat@width>\linewidth\linewidth\else\Gin@nat@width\fi}
\def\maxheight{\ifdim\Gin@nat@height>\textheight\textheight\else\Gin@nat@height\fi}
\makeatother
% Scale images if necessary, so that they will not overflow the page
% margins by default, and it is still possible to overwrite the defaults
% using explicit options in \includegraphics[width, height, ...]{}
\setkeys{Gin}{width=\maxwidth,height=\maxheight,keepaspectratio}
% Set default figure placement to htbp
\makeatletter
\def\fps@figure{htbp}
\makeatother

\KOMAoption{captions}{tableheading}
\makeatletter
\makeatother
\makeatletter
\makeatother
\makeatletter
\@ifpackageloaded{caption}{}{\usepackage{caption}}
\AtBeginDocument{%
\ifdefined\contentsname
  \renewcommand*\contentsname{Table of contents}
\else
  \newcommand\contentsname{Table of contents}
\fi
\ifdefined\listfigurename
  \renewcommand*\listfigurename{List of Figures}
\else
  \newcommand\listfigurename{List of Figures}
\fi
\ifdefined\listtablename
  \renewcommand*\listtablename{List of Tables}
\else
  \newcommand\listtablename{List of Tables}
\fi
\ifdefined\figurename
  \renewcommand*\figurename{Figure}
\else
  \newcommand\figurename{Figure}
\fi
\ifdefined\tablename
  \renewcommand*\tablename{Table}
\else
  \newcommand\tablename{Table}
\fi
}
\@ifpackageloaded{float}{}{\usepackage{float}}
\floatstyle{ruled}
\@ifundefined{c@chapter}{\newfloat{codelisting}{h}{lop}}{\newfloat{codelisting}{h}{lop}[chapter]}
\floatname{codelisting}{Listing}
\newcommand*\listoflistings{\listof{codelisting}{List of Listings}}
\makeatother
\makeatletter
\@ifpackageloaded{caption}{}{\usepackage{caption}}
\@ifpackageloaded{subcaption}{}{\usepackage{subcaption}}
\makeatother
\makeatletter
\@ifpackageloaded{tcolorbox}{}{\usepackage[skins,breakable]{tcolorbox}}
\makeatother
\makeatletter
\@ifundefined{shadecolor}{\definecolor{shadecolor}{rgb}{.97, .97, .97}}
\makeatother
\makeatletter
\makeatother
\makeatletter
\makeatother
\ifLuaTeX
  \usepackage{selnolig}  % disable illegal ligatures
\fi
\IfFileExists{bookmark.sty}{\usepackage{bookmark}}{\usepackage{hyperref}}
\IfFileExists{xurl.sty}{\usepackage{xurl}}{} % add URL line breaks if available
\urlstyle{same} % disable monospaced font for URLs
\hypersetup{
  pdftitle={HW4},
  pdfauthor={Ben Lawrence},
  colorlinks=true,
  linkcolor={blue},
  filecolor={Maroon},
  citecolor={Blue},
  urlcolor={Blue},
  pdfcreator={LaTeX via pandoc}}

\title{HW4}
\author{Ben Lawrence}
\date{}

\begin{document}
\maketitle
\ifdefined\Shaded\renewenvironment{Shaded}{\begin{tcolorbox}[interior hidden, sharp corners, borderline west={3pt}{0pt}{shadecolor}, boxrule=0pt, frame hidden, breakable, enhanced]}{\end{tcolorbox}}\fi

\hypertarget{homework-4}{%
\section{Homework 4}\label{homework-4}}

\hypertarget{exercise-1}{%
\subsection{Exercise 1}\label{exercise-1}}

The multiple linear regression model can be written as \(Y=X\beta+e\),
where \(\text{Var(e)}=\sigma^2I\) and \(I\) is the \(n \times n\)
identity matrix. The fitted values are given by

\[
\hat Y=X\hat \beta=X(X^TX)^{-1}X^TY=HY
\]

where \(H=X(X^TX)^{-1}X^T\)

(a) Show that \(HH^T=HH=H\). Note that a matrix that has this property
is called idempotent.

First we note the following properties of matrices

\[
\text{Let A, B, C be invertible matrices whose product }ABC \text{ exists}.\\
 \text{Then }(ABC)^T=C^T(AB)^T =C^TB^TA^T \text{and }(A^T)^{-1}=(A^{-1})^T
\]

Using these properties we obtain

\[
HH^T=X(X^TX)^{-1}X^T[X(X^TX)^{-1}X^T]^T \\
= X(X^TX)^{-1}X^T(X^T)^T[(X^TX)^{-1}]^TX^T \\
= X(X^TX)^{-1}X^TX[(X^TX)^T]^{-1}X^T \\
= X(X^TX)^{-1}X^TX(X^TX)^{-1}X^T=HH \checkmark \\
= X(X^TX)^{-1}(I)X^T \\
=X(X^TX)^{-1}X^T=H \checkmark
\]

(b) Show that \$ E(\hat Y)=X\beta\$

\[
 E(\hat Y)= E(HY)= E[H(X \beta+e)] \\
=H[\beta X+E(e)]=H[X \beta+ \mathbf{0}] \\
=HX \beta=X(X^TX)^{-1}X^TX\beta \\
=XI\beta=X\beta \checkmark
\]

(c) Show that \$ Var(\hat Y)=\sigma\^{}2H\$

\[
 Var(\hat Y)= Var(HY)= Var[H(X\beta+e)] \\
=H  Var(X\beta+e) \\
=H  Var(e)\\
=H\sigma^2I \\
=\sigma^2HI \\
=\sigma^2H \checkmark
\]

\hypertarget{exercise-2}{%
\subsection{Exercise 2}\label{exercise-2}}

In lecture 10 we showed the variance-covariance matrix for the
\((p+1)\times1\) vector, \(\hat \beta\), of least squares estimates is
given by \$ Var(\hat \beta)=\sigma\textsuperscript{2(X}TX)\^{}\{-1\}\$.
Derive the \(2\times2\) variance-covariance matrix for least squares
estimates, \(\hat\beta=(\hat\beta_0,\hat\beta_1)^T\), for simple linear
regression:

\[
 Var(\hat\beta)= 
\begin{pmatrix}
 Var(\hat\beta_0) &  Cov(\hat\beta_0,\hat\beta_1) \\
 Cov(\hat\beta_1,\hat\beta_0) &  Var(\hat\beta_1) \\
\end{pmatrix}
\]

Additionally, use your result to verify that \$
Var(\hat\beta\_0)=\sigma\^{}2(\frac{1}{n}+\frac{\bar x^2}{SXX})\$ and \$
Var(\hat\beta*1)=*\sigma\^{}2/SXX\$, where \$
SXX=\sum{i=1}\textsuperscript{n(x\_i-}\bar x)2\$. {[}Hint: it might be
useful to use the identity
\(\sum_{i=1}^n(x_i-\bar x)^2=\sum x_i^2-n\bar x^2\){]}

The \(X\) matrix for a simple linear regression with \(n\) points is
given by

\[
X=
\begin{bmatrix} 
1 & x_1 \\
1 & x_2 \\
\vdots & \vdots \\
1 & x_n
\end{bmatrix}
\]

Then we apply the general result to obtain the variance covariance
matrix

\[
\sigma^2(X^TX)^{-1}=\sigma^2
\left( 
\begin{bmatrix}
1  & x_1 \\
1 & x_2 \\
\vdots & \vdots \\
1 & x_n 
\end{bmatrix}
\begin{bmatrix} 
1 & 1 & \cdots & 1 \\
x_1 & x_2 & \cdots & x_n
\end{bmatrix}
\right)^{-1} \\
=\sigma^2 \left( 
\begin{bmatrix} 
1+1+\cdots+1 & x_1+x_2+\cdots+x_n \\
x_1+x_2+\cdots+x_n & x_1^2+x_2^2+\cdots+x_n^2
\end{bmatrix}
\right)^{-1} \\
= \sigma^2 \begin{bmatrix} 
n & \sum_{i=1}^nx_i \\
\sum_{i=1}^nx_i & \sum_{i=1}^nx_i^2
\end{bmatrix}^{-1 \\} \\
=\frac{\sigma^2}{n} 
\begin{bmatrix} 
1 & \bar x \\
\bar x & \frac{1}{n}\sum_{i=1}^nx_i^2
\end{bmatrix}^{-1} \\
=\sigma^2 \cdot\frac{1}{n(\frac{1}{n}\sum_{i=1}^nx_i^2-(\bar x)^2)} 
\begin{bmatrix} 
\frac{1}{n}\sum_{i=1}^nx_i^2 & -\bar x \\
-\bar x & 1
\end{bmatrix} \\
= \sigma^2 \cdot \frac{1}{SXX} \begin{bmatrix} 
\frac{1}{n}\sum_{i=1}^nx_i^2 & -\bar x \\
-\bar x & 1
\end{bmatrix}
\]

After applying the provided identity we obtain

\[
\frac{\frac{1}{n}\sum_{i=1}^nx_i^2}{SXX}=\frac{\frac{1}{n}(SXX+n\bar x)}{SXX}=\frac{1}{n}+ \frac{\bar x}{SXX}
\]

So our variance covariance matrix becomes,

\[
\begin{bmatrix} 
\sigma^2(\frac{1}{n}+\frac{\bar x}{SXX}) & \sigma^2(\frac{-\bar x}{SXX}) \\
\sigma^2(\frac{-\bar x}{SXX}) & \frac{\sigma^2}{SXX}
\end{bmatrix}
=\begin{bmatrix} 
 Var(\hat\beta_0) &  Cov(\hat\beta_0,\hat\beta_1) \\
 Cov(\hat\beta_1,\hat\beta_0) &  Var(\hat\beta_1)
\end{bmatrix} \checkmark
\]

\hypertarget{exercise-3}{%
\subsection{Exercise 3}\label{exercise-3}}

For this exercise use the \$ Boston\$ data set from the \$ MASS\$
package. Consider the multiple linear regression model with \$ medv\$ as
the response and \$ dis, rm, tax\$ and \$ chas\$ as predictor variables.

(a) In R, compute the vector of least squares estimates
\(\hat\beta=(X^TX)^{-1}X^TY\). Then verify that the results are the same
as the parameter estimates provided by the \$ lm()\$ function.

(b) In R, compute the variance-covariance matrix \$
Var(\hat\beta)=\sigma\textsuperscript{2(X}TX)\^{}\{-1\}\$ (plug in
\(\sigma^2= RSS/(n-p-1)\) as the estimate for \(\sigma^2\)). Then verify
that the square root of the diagonal entries of this matrix are the same
as the standard errors provided by the the \$ lm()\$ function.

\begin{Shaded}
\begin{Highlighting}[]
\FunctionTok{library}\NormalTok{(MASS)}
\FunctionTok{head}\NormalTok{(Boston)}
\end{Highlighting}
\end{Shaded}

\begin{verbatim}
     crim zn indus chas   nox    rm  age    dis rad tax ptratio  black lstat
1 0.00632 18  2.31    0 0.538 6.575 65.2 4.0900   1 296    15.3 396.90  4.98
2 0.02731  0  7.07    0 0.469 6.421 78.9 4.9671   2 242    17.8 396.90  9.14
3 0.02729  0  7.07    0 0.469 7.185 61.1 4.9671   2 242    17.8 392.83  4.03
4 0.03237  0  2.18    0 0.458 6.998 45.8 6.0622   3 222    18.7 394.63  2.94
5 0.06905  0  2.18    0 0.458 7.147 54.2 6.0622   3 222    18.7 396.90  5.33
6 0.02985  0  2.18    0 0.458 6.430 58.7 6.0622   3 222    18.7 394.12  5.21
  medv
1 24.0
2 21.6
3 34.7
4 33.4
5 36.2
6 28.7
\end{verbatim}

\hypertarget{part-a}{%
\subsubsection{Part a}\label{part-a}}

\begin{Shaded}
\begin{Highlighting}[]
\CommentTok{\#Create a matrix of predictors using the dis, rm, tax, and chas columns of the Boston dataset, including a column of 1s to account for beta\_0}
\NormalTok{X }\OtherTok{\textless{}{-}} \FunctionTok{as.matrix}\NormalTok{(}\FunctionTok{cbind}\NormalTok{(}\DecValTok{1}\NormalTok{, Boston[, }\FunctionTok{c}\NormalTok{(}\StringTok{\textquotesingle{}dis\textquotesingle{}}\NormalTok{, }\StringTok{\textquotesingle{}rm\textquotesingle{}}\NormalTok{, }\StringTok{\textquotesingle{}tax\textquotesingle{}}\NormalTok{, }\StringTok{\textquotesingle{}chas\textquotesingle{}}\NormalTok{)]))}

\CommentTok{\#Create a matrix of responses}
\NormalTok{Y }\OtherTok{\textless{}{-}} \FunctionTok{as.matrix}\NormalTok{(Boston[}\StringTok{\textquotesingle{}medv\textquotesingle{}}\NormalTok{])}

\CommentTok{\#Get the transpose of the predictors matrix}
\NormalTok{X\_t }\OtherTok{\textless{}{-}} \FunctionTok{t}\NormalTok{(X)}

\CommentTok{\#Apply the least{-}squares estimates formula}
\NormalTok{Beta }\OtherTok{\textless{}{-}} \FunctionTok{solve}\NormalTok{(X\_t }\SpecialCharTok{\%*\%}\NormalTok{ X) }\SpecialCharTok{\%*\%}\NormalTok{ X\_t }\SpecialCharTok{\%*\%}\NormalTok{ Y}
\NormalTok{Beta}
\end{Highlighting}
\end{Shaded}

\begin{verbatim}
             medv
1    -20.16720221
dis   -0.10656777
rm     7.88589232
tax   -0.01647039
chas   3.87901205
\end{verbatim}

Now we use the \$ lm\$ function and see if our results agree

\begin{Shaded}
\begin{Highlighting}[]
\NormalTok{lm1 }\OtherTok{\textless{}{-}} \FunctionTok{lm}\NormalTok{(medv }\SpecialCharTok{\textasciitilde{}}\NormalTok{ dis }\SpecialCharTok{+}\NormalTok{ rm }\SpecialCharTok{+}\NormalTok{ tax }\SpecialCharTok{+}\NormalTok{ chas, }\AttributeTok{data=}\NormalTok{Boston)}
\FunctionTok{summary}\NormalTok{(lm1)}
\end{Highlighting}
\end{Shaded}

\begin{verbatim}

Call:
lm(formula = medv ~ dis + rm + tax + chas, data = Boston)

Residuals:
    Min      1Q  Median      3Q     Max 
-19.878  -3.104  -0.520   2.384  42.086 

Coefficients:
              Estimate Std. Error t value Pr(>|t|)    
(Intercept) -20.167202   2.917308  -6.913 1.45e-11 ***
dis          -0.106568   0.152775  -0.698 0.485785    
rm            7.885892   0.402090  19.612  < 2e-16 ***
tax          -0.016470   0.001939  -8.494 2.29e-16 ***
chas          3.879012   1.073059   3.615 0.000331 ***
---
Signif. codes:  0 '***' 0.001 '**' 0.01 '*' 0.05 '.' 0.1 ' ' 1

Residual standard error: 6.034 on 501 degrees of freedom
Multiple R-squared:  0.573, Adjusted R-squared:  0.5696 
F-statistic: 168.1 on 4 and 501 DF,  p-value: < 2.2e-16
\end{verbatim}

And we can observe that they do.

\hypertarget{part-b}{%
\subsubsection{Part b}\label{part-b}}

\begin{Shaded}
\begin{Highlighting}[]
\CommentTok{\#First we obtain the factors necessary to calculate the variance of predicted values}
\NormalTok{RSS }\OtherTok{\textless{}{-}} \FunctionTok{sum}\NormalTok{((}\FunctionTok{residuals}\NormalTok{(lm1))}\SpecialCharTok{\^{}}\DecValTok{2}\NormalTok{)}
\NormalTok{n }\OtherTok{\textless{}{-}} \FunctionTok{nrow}\NormalTok{(Boston)}

\CommentTok{\#Calculate sigma squared manually}
\NormalTok{sigma\_squared }\OtherTok{\textless{}{-}}\NormalTok{ RSS }\SpecialCharTok{/}\NormalTok{ (n }\SpecialCharTok{{-}} \DecValTok{4} \SpecialCharTok{{-}} \DecValTok{1}\NormalTok{)}

\CommentTok{\#Now we apply the given formula to obtain the covariance matrix}
\NormalTok{Cov }\OtherTok{\textless{}{-}}\NormalTok{ sigma\_squared }\SpecialCharTok{*} \FunctionTok{solve}\NormalTok{(X\_t }\SpecialCharTok{\%*\%}\NormalTok{ X)}

\CommentTok{\#From the matrix we isolate the diagonal entries and square root them to obtain the standard errors}
\NormalTok{Err }\OtherTok{\textless{}{-}} \FunctionTok{sqrt}\NormalTok{(}\FunctionTok{diag}\NormalTok{(Cov))}
\NormalTok{Err}
\end{Highlighting}
\end{Shaded}

\begin{verbatim}
         1        dis         rm        tax       chas 
2.91730838 0.15277528 0.40208969 0.00193904 1.07305924 
\end{verbatim}

Indeed the manually calculated standard errors and errors reported by
the \$ lm\$ function are the same \(\checkmark\)



\end{document}
